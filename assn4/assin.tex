

\documentclass[12pt]{article}


%\usepackage{scicite}



\usepackage{times}


\topmargin 0.0cm
\oddsidemargin 0.2cm
\textwidth 16cm 
\textheight 21cm
\footskip 1.0cm



\newenvironment{sciabstract}{%
\begin{quote} \bf}
{\end{quote}}



\renewcommand\refname{References and Notes}


\newcounter{lastnote}
\newenvironment{scilastnote}{%
\setcounter{lastnote}{\value{enumiv}}%
\addtocounter{lastnote}{+1}%
\begin{list}%
{\arabic{lastnote}.}
{\setlength{\leftmargin}{.22in}}
{\setlength{\labelsep}{.5em}}}
{\end{list}}




\title{{\it Assignment 2\/} } 




\author
{Satyanand\\
\normalize{\textbf{14EC10049}}
}


\date{}






\begin{document} 

% Double-space the manuscript.

\baselineskip24pt

% Make the title.

\maketitle 

\section*{Problem 1}


This problem is popularly known as the josephus problem.

\subsection*{Algorithm}
\begin{itemize}

    
\item Alternating elimination means one of every two participants is eliminated. This is halving, and suggests powers of two are involved. Let’s first explore this with the special case where the number of participants is a power of two, since powers of two halve neatly into powers of two.
\item The elimination process works like this: the first pass starts at person 1 and proceeds clockwise, and each new pass starts every time person 1 is reached. The people eliminated on a pass are crossed out, and are marked to indicate the order in which they were eliminated. Eliminated people are then omitted in subsequent diagrams

\end{itemize}
\subsection*{ Analysis}
Regardless of the number of participants n, person 1 survives the first pass. Since n is even, as every positive power of two is, person 1 survives the second pass as well. In the first pass, the process goes like this: person 1 is skipped, person 2 is eliminated, person 3 is skipped, person 4 is eliminated, … , person n-1 is skipped, person n is eliminated. The second pass starts by skipping person 1.

As long as the number of participants per pass is even, as it will be for a power of two starting point, the same pattern is followed: person 1 is skipped each time. Therefore, for any power of two, person 1 always wins.

\subsection*{Proof by induction}
You can also show that person 1 is the winner using an inductive proof (for details see Miguel Lerma’s proof of the Josephus problem). Compared to the argument above, induction works in the opposite direction; that is, it builds up to a more complicated problem from a simpler one.

For n = 21 = 2 participants, the base case, it’s easy to see that person 1 is the winner. For the induction hypothesis, assume person 1 is the winner for n = 2m. Show person 1 is the winner for n = 2m+1.

When n = 2m+1, 2m people — all the even numbered people — are eliminated in the first pass, leaving 2m people — all the odd numbered people — remaining. By the induction hypothesis, person 1 is the winner of the n = 2m remaining people, and thus the winner among all n = 2m+1 people. Therefore, for any power of two, person 1 always wins.

Here the number of levels of recursion(recursively calling union) is lg(n).At each level we are taking union by looping over the available buildings. Each building is traversed only once at each level hence the time complexity of each level is O(n).Hence the overall complexity of the algorithm is O(nlgn).



\subsection*{When the Number of Participants is NOT a Power of Two}
When the number of participants is not a power of two, we know this much: person 1 can’t be the winner. This is because at least one pass will have an odd number of participants. Once the first odd participant pass is complete, person 1 will be eliminated at the start of the next pass.

So is there an easy way to determine who is the winner? Let’s step back and take a closer look at the elimination process in the 13-person example:

\subsection*{Analysis}
Powers of two come into play here, but you have to change your perspective to see them. They don’t occur on pass boundaries — they span them. At some point, during pass 1, the number of participants remaining becomes a power of two. In this example, that occurs when 5 of the 13 people are eliminated, leaving an 8 person problem: 11, 12, 13, 1, 3, 5, 7, 9. This means person 11, the first person in the new power of two circle, wins.

\subsection*{Result}
We can solve both cases — in other words, for an arbitrary number of participants — using a little math.

Write n as n = 2m + k, where 2m is the largest power of two less than or equal to n. k people need to be eliminated to reduce the problem to a power of two, which means 2k people must be passed over. The next person in the circle, person 2k + 1, will be the winner. In other words, the winner w is w = 2k + 1.




\subsection*{Solving by mapping}

We can solve this progrmatically by using sort of divide and conquer.We first eleminate all alternate persons.Now the number of persons left can be mapped to pseudo numbers from 1 to floor(n/2) by using the equations n=2k-1(if n is even) or n=2k+1(if n is odd).Now the persons to be eliminated can be obtained from these pseudo numbers(persons corresponding to even pseudo numbers are to be eliminated).This can be used recursively.At each recursive step the mapping change.But the mapping will always be a linear function of order one.Hence we just have to maintain 2 variables a and b in the mapping n=ak+b.

\subsection*{Complexity analysis}
The prgrammatic method of obtaining the sequence takes linear time  in n i.e. O(n) because we are evaluating mapping for each person only once. The evaluation of mapping requires conatant time.





\end{document}



















